\documentclass[]{beamer}
%\documentclass[notes]{beamer}       % print frame + notes
%\documentclass[notes=only]{beamer}   % only notes
%\documentclass{beamer}              % only frames
%\documentclass[handout]{beamer}
\usepackage{tikz}
\usepackage{amsmath}
\usepackage{algorithm2e}

\usetheme{Dresden}%%%%% developer's preference - may change based on preferences

%%%%%% UMass official color: https://www.umass.edu/brand/elements/color
\definecolor{UMassAmherst}{rgb}{0.533 0.11 0.11}
\usecolortheme[named=UMassAmherst]{structure}

\title{Grafos\\B\'usqueda}
\author{MSc Edson Ticona Zegarra}
\institute{Taller avanzado 2025}
\date{}

%%%%%% obtained from: https://www.umass.edu/brand/elements/wordmarks-seal-and-spirit-marks
%%%%%% logos of other departments can also be obtained from the above link. Otherwise, consult your department website.

\begin{document}

\maketitle

\begin{frame}{Contenido}
\tableofcontents
\end{frame}

\section{Introducci\'on}
\begin{frame}{Contenido}
\tableofcontents[currentsection]
\end{frame}

\begin{frame}{Grafo}
  \begin{itemize}
    \item Un grafo es un par $G=(V,E)$ de conjuntos tal que $E \subset V^2$
      \pause
    \item As\'i, todo elemento de $E$ es un subconjunto de 2 elementos de $V$. Los elementos de $V$ son los v\'ertices y los elementos de $E$ son sus aristas.
      \pause
    \item Los v\'ertices de $G$ son denotados por $V(G)$ y sus aristas por $E(G)$
      \pause
    \item Un v\'ertice $v$ es incidente con una arista $e$ si $v \in e$
      \pause
    \item El conjunto de v\'ertices adyacentes a $v$ se denota por $E(v)$
      \pause
    \item El grado $d_G(v)$ o $d(v)$ de un v\'ertice es el n\'umero $|E(v)|$ de aristas en $v$ 
      \pause
    \item Dos v\'ertices $u, v$ son adyacentes si existe una arista $e = \{u,v\}$
  \end{itemize}
\end{frame}

\begin{frame}{Caminos y ciclos}
  \begin{itemize}
    \item Un \textit{camino} es un grafo no vacio $P=(V,E)$, tal que $V=\{x_0, x_1, ..., x_k \}$ y $E = \{x_0x_1, x_1x_2, ..., x_{k-1}x_k \}$
      \pause
    \item Donde todo $x_i$ es diferente. El n\'umero de aristas de un camino es su tama\~no o longitud
      \pause
    \item Si $P = x_0, ..., x_{k-1}$ es un camino y $k\geq 3$, entonce el grafo $C = P + x_{k-1}x_0$ es llamado de \textit{ciclo}
  \end{itemize}
\end{frame}

\begin{frame}{Conectividad}
  \begin{itemize}
    \item Se dice que un grafo $G$ est\'a \textit{conectado} si existe un camino entre cualquier par de v\'ertices
      \pause
    \item Un subgrafo maximal conectado de G es llamado de \textit{componente} de G
  \end{itemize}
\end{frame}

\begin{frame}{\'Arboles y Florestas}
  \begin{itemize}
    \item Un grafo ac\'iclico es llamado de \textit{floresta}
      \pause
    \item Un floresta conectada es llamada de \textit{\'arbol}
      \pause
    \item Es decir, una floresta es un grafo cuyos componentes son \'arboles.
      \pause
    \item Los v\'ertices de grado 1 de un \'arbol son \textit{hojas (leaves)}
  \end{itemize}
\end{frame}

\begin{frame}{\'Arboles y Florestas: definiciones}
  \begin{itemize}
    \item Todo par de v\'ertices de un \'arbol $T$ est\'a conectado por un camino \'unico 
      \pause
    \item T est\'a minimamente conectado, es decir, $T$ est\'a conectado pero $T-e$ esta desconectado para cualquier $e\in T$
      \pause
    \item T is maximalmente ac\'iclico, es decir, $T$ \textbf{no} tiene ciclos pero $T + xy$ tiene ciclos para cualquier par de v\'ertices no adyacentes $x, y \in T$
  \end{itemize}
\end{frame}

\begin{frame}{\'Arboles y Florestas: definiciones}
  \begin{itemize}
    \item Sea $T$ un \'arbol enraizado en $r$ y sea $x$ un v\'ertice cualquier. Se llama de \textit{ancestro} a todo v\'ertice $y$ tal que $y$ se encuentra en el camino entre $r$ y $x$
      \pause
    \item As\'i mismo, se llama a $x$ de \textit{decendiente} de $y$
  \end{itemize}
\end{frame}

\begin{frame}{Cortes}
  \begin{itemize}
    \item Dado un conjunto $V$, hacer una \textit{partici\'on} de $V$ es agrupar sus elementos en subconjuntos no vacios tal que todo elemento est\'a incluido exactamente en un subconjunto
      \pause
    \item Si $V_1$ y $V_2$ es una partici\'on de $V$, el conjunto de aristas que van de $V_1$ a $V_2$ se conoce como un \textit{corte}
      \pause
    \item Note que si $V_1 = { v }$, entonces el corte es denotado por $E(v)$
  \end{itemize}
\end{frame}

\section{Representaci\'on de grafos}
\begin{frame}{Representaci\'on de grafos}
  \begin{itemize}
    \item Matriz de adyacencia: \texttt{int AM[n][n]}
      \pause
    \item Lista de adyacencia: \texttt{vector<vector<pair<int,int>> al}. Cada par guarda el \'indice y el peso de la arista
      \pause
    \item Lista de aristas: \texttt{vector<tuple<int, int, int>> el}. Cada tupla guarda el peso, y las aristas
  \end{itemize}
\end{frame}

\begin{frame}{Grafos impl\'icitos}
  \begin{itemize}
    \item Algunos grafos no son almacenados directamene, sino que son generados expl\'icitamente en alguna operaci\'on
      \pause
    \item Estos grafos se conocen como grafos \textit{impl\'icitos}
      \pause
    \item Por ejemplo, navegar una grilla 2D, un tablero de ajedrez, etc
  \end{itemize}
\end{frame}

\section{B\'usqueda en grafos}
\subsection{BFS}
\begin{frame}{Breadth First Search: B\'usqueda en amplitud}
  \begin{algorithm}[H]
    %\SetKwInOut{Input}{input}\SetKwInOut{Output}{output}
    %\Input{$G$ es el grafo y $x$ el v\'ertice buscado.}
    %\Output{$0$ si $x \in V$, $1$ caso contrario}
    %\BlankLine
    {$ toVisit.push(u) $} \tcp*{$toVisit$ es una cola, nodo $u$ arbitrario}
    \While{$ !toVisit.empty() $}
    {
      $u = toVisit.front() $ \;
      \For{$v \in E(u)$}
      {
        % \tcc{$Adj(u)$ adyacentes a $u$}
        % \tcc{colorear los v\'ertices}
        \If{$ !visited(v) $}
        {
          {$ toVisit.push(v)  $} \;
          { dist[v] = dist [u] + 1 } \;
          { parent[v] = $u$ } \;
          { mark $u$ as visited } \;
        }
      }
      { $toVisit.pop()$ }
    }
  \end{algorithm}
\end{frame}

\begin{frame}{Breadth First Search: B\'usqueda en amplitud}
  \begin{itemize}
    \item La complejidad temporal es $O(|V| + |E|)$
      \pause
    \item La complejidad espacial es $O(|V|)$
  \end{itemize}
\end{frame}

\begin{frame}{Caminos m\'inimos}
  \begin{itemize}
    \item Para grafos sin pesos, BFS encuentra la distancia m\'inima $\delta(s, v)$ entre $s$ y $v$, definiendo la distancia como el tama\~no del camino entre $s$ y $v$, $\forall v \in V(G)$
      \pause
    \item En el seudoc\'odigo, $\delta(s, u)$ ser\'a $dist[u]$ donde $s$ es el v\'ertice de inicio
  \end{itemize}
\end{frame}

\begin{frame}{Breadth-first Tree}
  \begin{itemize}
    \item El \'orden de visita de los v\'ertices define un \textit{breadth-first tree}. 
      \pause
    \item Definimos el breadth-first tree como $G_{\pi} = (V_{\pi}, E_{\pi})$ tal que $V_{\pi} = \{ v\in V : v.parent \neq NIL \} \cup \{s\}$ y $E_{\pi} = \{ (v.parent, v) : v\in V_{\pi} - \{s\} \}$
      \pause
    \item La estructura \textit{parent} contiene el breadth-first tree enraizado en $s$
      \pause
    \item $G_{\pi}$ consiste de los v\'ertices alcanzables desde $s$ y a su vez contiene un camino \'unico de $s$ a $v$, siendo el camino m\'inimo
  \end{itemize}
\end{frame}

\begin{frame}{Breadth-first Tree: Imprimir camino}
  \begin{algorithm}[H]
    \SetKwInOut{Input}{input}\SetKwInOut{Output}{output}
    \Input{$G, s, v$ : G grafo, camino de $s$ a $v$}
    \If{ $ v == s $ }
    {
      { print $s$ }
    } \Else {
      \If{ $v.parent == NIL $ }
      {
        { print "no path" }
      } \Else {
        { PrintPath($G, s, v.parent$) } \\
        { print $v$ }
      }
    }
  \end{algorithm}
\end{frame}

\subsection{DFS}
\begin{frame}{Depth First Search: B\'usqueda en profundidad}
  \begin{algorithm}[H]
    %\SetKwInOut{Input}{input}\SetKwInOut{Output}{output}
    %\Input{$G$ es el grafo y $x$ el v\'ertice buscado.}
    %\Output{$0$ si $x \in V$, $1$ caso contrario}
    %\BlankLine
    {$ toVisit.push(u) $} \tcp*{$toVisit$ es una pila y se comienza en un v\'ertice arbitrario $u$}
    \While{$ !toVisit.empty() $}
    {
      $u = toVisit.front() $
      \tcc{$Adj(u)$ adyacentes a $u$}
      \For{$v \in Adj(u)$}
      {
        \tcc{colorear los v\'ertices}
        \If{$ !visited(v) $}
        {
          {$ toVisit.push(v)  $} \;
          { parent[v] = $u$ } \;
          { mark $u$ as visited } \;
        }
      }
      { $toVisit.pop()$ }
    }
  \end{algorithm}
\end{frame}


\begin{frame}{Depth First Search (DFS): B\'usqueda en profundidad}
  \begin{algorithm}[H]
    %\SetKwInOut{Input}{input}\SetKwInOut{Output}{output}
    %\Input{$A$ es un conjunto de $n$ n\'umeros y $x$ es el n\'umero buscado.}
    %\Output{$0$ si $x \in A$, $1$ caso contrario}
    %\BlankLine
    mark $u$ as visited \\
    $u.d \gets time++$ \\
    \For{$v \in Adj(u)$}
    {
      \If{$ !visited(v) $}
      {
        $ parent[v] = u $ \\
        $ DFS(v) $
      }
    }
    $u.f \gets ++time$ \\
  \end{algorithm}
\end{frame}

\begin{frame}{Depth First Search: B\'usqueda en amplitud}
  \begin{itemize}
    \item $u.d$ es el tiempo de \textit{discovery}
      \pause
    \item $u.f$ es el tiempo de \textit{finalizaci\'on}
      \pause
    \item La complejidad temporal es $O(|V| + |E|)$
      \pause
    \item La complejidad espacial es $O(|V|)$
  \end{itemize}
\end{frame}

\begin{frame}{Depth First Search (DFS): Clasificaci\'on de aristas}
  \begin{itemize}
    \item \textbf{Tree edges}: aristas en el grafo $G_{\pi}$, siendo $G_{\pi}$ el \textit{depth-first tree}
      \pause
    \item \textbf{Back edges}: aristas conectando un v\'ertice $u$ a un ancestro $v$ en un \textit{depth-first tree}
      \pause
    \item \textbf{Forward edges}: aristas conectando un v\'ertice $u$ a un decendiente $v$ en un \textit{depth-first tree}
      \pause
    \item \textbf{Cross edges}: el resto de aristas
  \end{itemize}
\end{frame}

\begin{frame}{Orden topol\'ogico}
  \begin{itemize}
    \item Un ordenamiento topol\'ogico se define para un grafo dirigido ac\'iclico (DAG en ingl\'es)
      \pause
    \item Es un ordenamiento lineal de los v\'ertices tal que $u$ aparece antes de $v$ si existe una arista dirigida $(u,v)$. 
      \pause
    \item Podemos usar el DFS para hallar los tiempos de finalizaci\'on y usarlos de referencia para el \'orden
  \end{itemize}
\end{frame}

\begin{frame}{Ordenamiento topol\'ogico}
  \begin{algorithm}[H]
    %\SetKwInOut{Input}{input}\SetKwInOut{Output}{output}
    %\Input{$G$ es el grafo y $x$ el v\'ertice buscado.}
    %\Output{$0$ si $x \in V$, $1$ caso contrario}
    %\BlankLine
    \textsc{$DFS(G)$} \\
    \textsc{Sort} en base a u.f
  \end{algorithm}
\end{frame}

\begin{frame}{Ordenamiento topol\'ogico: Algoritmo de Kahn}
  \begin{algorithm}[H]
    %\SetKwInOut{Input}{input}\SetKwInOut{Output}{output}
    %\Input{$G$ es el grafo y $x$ el v\'ertice buscado.}
    %\Output{$0$ si $x \in V$, $1$ caso contrario}
    %\BlankLine
    \For { $u \in V(G)$ }
    {
      \If { $d(u) == 0$ }
      {
        {$ toVisit.push(u) $} \tcp*{$toVisit$ es una cola}
      }
    }

    \While { $! toVisit.empty() $ }
    {
      $ u \gets toVisit.pop() $ \\
      \For{ $v \in E(u) $ }
      {
        { $d(v) \gets d(v)-1 $} \\
        \If{ $d(v) > 0 $ } 
        {
          continue
        }
        $ toVisit.push(v) $
      }
    }
  \end{algorithm}
\end{frame}

\begin{frame}{Componentes conectados}
  \begin{itemize}
    \item Podemos utilizar el DFS o BFS para hallar los compoenentes conectados
      \pause
    \item Ambos van a recorrer cada componente por cada llamada
  \end{itemize}
\end{frame}

\begin{frame}{Verificar grafos bipartitos}
  \begin{itemize}
    \item Se dice que un grafo $G=(V,E)$, si $V$ admite una partici\'on en dos tal que toda arista tenga un extremo en cada partici\'on. 
      \pause
    \item En general se dice que un grafo es $r$-partito si admite $r$ particiones tal que toda arista tenga un extremo en alguna partici\'on
      \pause
    \item V\'ertices en la misma partici\'on no pueden ser adyacentes
      \pause
    \item Podemos usar BFS o DFS para ello, coloreando los v\'ertices de manera alternada
  \end{itemize}
\end{frame}

\begin{frame}{Verificar ciclos}
  \begin{itemize}
    \item Podemos utilizar un DFS para verificar existencia de ciclos
      \pause
    \item Si se encuentra un back edge es una prueba de que existe un ciclo
  \end{itemize}
\end{frame}

\end{document}
