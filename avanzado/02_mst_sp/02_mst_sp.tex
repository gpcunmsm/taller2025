\documentclass[]{beamer}
%\documentclass[notes]{beamer}       % print frame + notes
%\documentclass[notes=only]{beamer}   % only notes
%\documentclass{beamer}              % only frames
%\documentclass[handout]{beamer}
\usepackage{tikz}
\usepackage{amsmath}
\usepackage{algorithm2e}

\usetheme{Dresden}%%%%% developer's preference - may change based on preferences

%%%%%% UMass official color: https://www.umass.edu/brand/elements/color
\definecolor{UMassAmherst}{rgb}{0.533 0.11 0.11}
\usecolortheme[named=UMassAmherst]{structure}

\title{Grafos\\Minimum Spanning Tree \& Shortest Paths}
\author{MSc Edson Ticona Zegarra}
\institute{Taller avanzado 2025}
\date{}

%%%%%% obtained from: https://www.umass.edu/brand/elements/wordmarks-seal-and-spirit-marks
%%%%%% logos of other departments can also be obtained from the above link. Otherwise, consult your department website.

\begin{document}

\maketitle

\begin{frame}{Contenido}
\tableofcontents
\end{frame}

\section{Minimum Spanning Tree}
\begin{frame}{Contenido}
\tableofcontents[currentsection]
\end{frame}

\begin{frame}{Grafo de expansi\'on}
  \begin{itemize}
    \item Dado un grafo $G=(V,E)$ y un subgrafo $G'=(V',E')$, tal que $G' \subset G$, se dice que $G'$ es un subgrafo de expansi\'on si todos los v\'ertices de $V'$ se expanden a todo $G$, es decir si $V'=V$
      \pause
    \item Llamamos a un grafo $G$ de \textit{edge-maximal} con cierta propiedad si $G$ cumple con la propiedad pero el grafo $G+xy$ no, para cualquier par de v\'ertices $x, y \in G$
      \pause
    \item En general hablamos de \textit{maximal} o \textit{minimal} con cierta propiedad, se refiere a la relaci\'on de subgrafo
  \end{itemize}
\end{frame}

\begin{frame}{\'Arbol de expansi\'on m\'inimo}
  \begin{itemize}
    \item Llamamos de \'arbol de expansi\'on a todo subgrafo $T \in G$ tal que $T$ es un \'arbol y $T$ es un subgrafo de expansi\'on
      \pause
    \item Para un grafo con pesos, entre todos los posibles \'arboles de expansi\'on, se llama a $T$ de \'arbol de expansi\'on m\'inimo si la suma de los pesos de las aristas de $T$ es la menor de todas
      \pause
    \item Formalmente, $$\min \sum_{e \in E} w(e)$$
      \pause
    \item donde $w(e)$ representa el peso de la arista $e$
  \end{itemize}
\end{frame}

\begin{frame}{Minimum Spanning Tree (MST)}
  \begin{itemize}
    \item El problema del \'arbol de expansi\'on m\'inimo, en ingl\'es \textit{minimum spanning tree (MST)} admite soluci\'on por un algoritmo greedy
      \pause
    \item La idea es construir el MST agregando arista por arista, siempre que sea una arista \textit{segura} de peso m\'inimo
      \pause
    \item Una arista segura es aquella que mantiene la propiedad
  \end{itemize}
\end{frame}

\begin{frame}{Algoritmo de Prim}
  \begin{itemize}
    \item El algoritmo de Prim inicia de un v\'ertice arbitrario, dicho v\'ertice provee diversas opciones en cuanto a cu\'al arista agregar a continuaci\'on
      \pause
    \item Siempre se agrega la menor disponible al componente creado hasta el momento
  \end{itemize}
\end{frame}

\begin{frame}{Prim}
  \begin{algorithm}[H]
    \SetKwInOut{Input}{input}\SetKwInOut{Output}{output}
    \Input{$G=(V,E)$ es el grafo}
    \Output{$T$ es el MST}
    %\BlankLine
    % {$ T.push(u) $} \tcp*{$T$ es el MST, nodo $u$ arbitrario}
    { $ Q.insert(V) $ }
    \For{$ v \in V $}
    {
      \If{$e.isSafe()$}
      {
        {$ T.push(e)  $}
      }
    }
  \end{algorithm}
\end{frame}

\begin{frame}{Prim}
  \begin{itemize}
    \item Complejidad $O()$
  \end{itemize}
\end{frame}

\begin{frame}{Algoritmo de Kruskal}
  \begin{itemize}
    \item El algoritmo de Kruskal inicia por la arista de menor peso
      \pause
    \item Se continue con la siguiente arista de menor peso, si dicha arista no pertenece al mismo componente se crea otro componente
      \pause
    \item Eventualmente una arista va a unir un par de componentes desconectados
  \end{itemize}
\end{frame}

\begin{frame}{Kruskal}
  \begin{algorithm}[H]
    %\SetKwInOut{Input}{input}\SetKwInOut{Output}{output}
    %\Input{$G$ es el grafo y $x$ el v\'ertice buscado.}
    %\Output{$0$ si $x \in V$, $1$ caso contrario}
    %\BlankLine
    {$Sort(E)$} \\
    \For{$ e \in E $}
    {
      \If{$e.isSafe()$}
      {
        {$ T.push(e)  $}
      }
      { $toVisit.pop()$ }
    }
  \end{algorithm}
\end{frame}

\begin{frame}{Kruskal}
  \begin{itemize}
    \item Complejidad $O()$
  \end{itemize}
\end{frame}

\subsection{Union-Find Disjoint Sets}
\begin{frame}{Conjuntos Disjuntos}
  \begin{itemize}
    \item Estructura de datos para manejar conjuntos disjuntos de manera eficiente
      \pause
    \item 
  \end{itemize}
\end{frame}

\section{Caminos m\'inimos}
\begin{frame}{Caminos m\'inimos}
  \begin{itemize}
    \item Dado un grafo con pesos, se busca minimizar la suma de los pesos de las aristas del camino entre $s$ y $t$
      \pause
    \item Formalmente, $$ \min \sum_{e \in P} w(e)$$
      \pause
    \item Tal que $P$ es un camino entre $s$ y $t$
  \end{itemize}
\end{frame}

\begin{frame}{Relajaci\'on}
  \begin{itemize}
    \item Dadar la relajaci\'on
  \end{itemize}
\end{frame}

\begin{frame}{Algoritmo de Bellman-Ford}
  \begin{itemize}
    \item Dado un grafo con pesos, se busca minimizar la suma de los pesos de las aristas del camino entre $s$ y $t$
      \pause
    \item Formalmente, $$ \min \sum_{e \in P} w(e)$$
      \pause
    \item Tal que $P$ es un camino entre $s$ y $t$
  \end{itemize}
\end{frame}

\end{document}
