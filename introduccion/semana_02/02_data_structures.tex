\documentclass[]{beamer}
%\documentclass[notes]{beamer}       % print frame + notes
%\documentclass[notes=only]{beamer}   % only notes
%\documentclass{beamer}              % only frames
%\documentclass[handout]{beamer}
\usepackage{tikz}

\usepackage{algorithm2e}
\usepackage{listings}

\usetheme{Dresden}%%%%% developer's preference - may change based on preferences

%%%%%% UMass official color: https://www.umass.edu/brand/elements/color
\definecolor{UMassAmherst}{rgb}{0.533 0.11 0.11}
\usecolortheme[named=UMassAmherst]{structure}

\title{Algoritmos}
\subtitle{Estructuras de Datos}
\author{Grupo de Programaci\'on}
\institute{Campamento de Programaci\'on}
\date{}

%%%%%% obtained from: https://www.umass.edu/brand/elements/wordmarks-seal-and-spirit-marks
%%%%%% logos of other departments can also be obtained from the above link. Otherwise, consult your department website.

% \titlegraphic{\includegraphics[width=0.5in]{logo_unmsm.png}}

\begin{document}

\maketitle

\begin{frame}{Contenido}
\tableofcontents
\end{frame}

\section{Estructuras de Datos}
\begin{frame}{Contenido}
\tableofcontents[currentsection]
\end{frame}

\begin{frame}{Estructuras de Datos}
  \begin{itemize}
    \item Una estructura de datos es una forma de organizar data para facilitar su posterior uso
      \pause
    \item Toda estructura de datos define operaciones como inserci\'on, remoci\'on, b\'usqueda, etc
      \pause
    \item Las estructuras de datos m\'as b\'asicas son los arreglos y las matrices (arreglos bidimensionales)
      \pause
    \item El objetivo es que sean un soporte para el dise\~no de algoritmos eficientes
      \pause
    \item C++ provee los llamados \textit{contenedores} que implementan varias estructuras de datos, hay una buena cantidad de estructuras de datos implementadas, las m\'as especializadas pueden ser construidas tomando estas de base.
  \end{itemize}
\end{frame}

\subsection{Estructuras de Datos Elementales}
\begin{frame}{Contenido}
\tableofcontents
\end{frame}

\begin{frame}{Vectores}
  \begin{itemize}
    \item Consideramos como \texttt{vectores} a arreglos con memoria din\'amica
      \pause
    \item Memoria est\'atica: se conoce el tama\~no del arreglo en tiempo de compilaci\'on (al momento de compilar el programa)
      \pause
    \item Memoria din\'amica: \textbf{no} se conoce el tama\~no del arreglo en tiempo de compilaci\'on, sino solo en tiempo de ejecuci\'on (al momento de correr el programa)
  \end{itemize}
\end{frame}

\begin{frame}{Vectores}
  \lstinputlisting[caption=Vectores en C++, label={lst:listing-cpp}, language=C++, basicstyle=\fontsize{8}{9}\selectfont]{vector\_sample.cpp}
\end{frame}

\begin{frame}{Pilas}
  \begin{itemize}
    \item Es un contenedor que almacena un conjunto de elementos, siendo una memoria tipo LIFO (Last-In First-Out)
      \pause
    \item Solo permite acceso al \'ultimo elemento agregado
      \pause
    \item Se puede pensar como una forma de apilar platos, uno encima de otro, de esta manera solo se puede retirar el \'ultimo que fue agregado
  \end{itemize}
\end{frame}

\begin{frame}{Pilas}
  \lstinputlisting[caption=Vectores en C++, label={lst:listing-cpp}, language=C++, basicstyle=\fontsize{8}{9}\selectfont]{stack\_sample.cpp}
\end{frame}

\begin{frame}{Colas}
  \begin{itemize}
    \item Es un contenedor que almacena un conjunto de elementos, siendo una memoria tipo FIFO (First-In First-Out)
      \pause
    \item Solo permite acceso al primer elemento agregado
      \pause
    \item Se puede pensar como una cola del banco, donde cada usuario que llega tiene que esperar que los que llegaron antes son atentidos primero salir de la cola
  \end{itemize}
\end{frame}

\begin{frame}{Colas}
  \lstinputlisting[caption=Vectores en C++, label={lst:listing-cpp}, language=C++, basicstyle=\fontsize{8}{9}\selectfont]{queue\_sample.cpp}
\end{frame}

\begin{frame}{Colas de prioridad}
  \begin{itemize}
    \item Es un contenedor que almacena un conjunto de elementos cada uno con cierta prioridad, siendo una memoria tipo FIFO (First-In First-Out)
      \pause
    \item Solo permite acceso al elemento con mayor prioridad
      \pause
    \item Se puede pensar como una cola del banco con diferentes tipos de usuarios (clientes regulares, clientes vip, no clientes, ec), donde cada usuario es atendido seg\'un su prioridad
  \end{itemize}
\end{frame}

\begin{frame}{Colas de prioridad}
  \lstinputlisting[caption=Vectores en C++, label={lst:listing-cpp}, language=C++, basicstyle=\fontsize{8}{9}\selectfont]{priority\_queue\_sample.cpp}
\end{frame}

\begin{frame}{Sets}
  \begin{itemize}
    \item Sets o conjuntos representan un conjunto de elementos sin repetici\'on
      \pause
    \item Adicionalmente, los sets preservan el orden de los elementos
      \pause
    \item Mantener el orden implica m\'as esfuerzo computacional, por tanto se pierde un poco de eficiencia
      \pause
    \item Si no se desea mantener el orden y solo se requiere unicidad de los elementos se puede usar \texttt{unordered\_set}
  \end{itemize}
\end{frame}

\begin{frame}{Sets}
  \lstinputlisting[caption=Vectores en C++, label={lst:listing-cpp}, language=C++, basicstyle=\fontsize{8}{9}\selectfont]{set\_sample.cpp}
\end{frame}

\begin{frame}{Listas enlazadas}
  \begin{itemize}
    \item Las listas proveen acceso solo al primer elemento y al \'ultimo, se debe iterar para acceder al resto
      \pause
    \item Es decir, a diferencia de los vectores, no provee acceso aleatorio (Random Access)
      \pause
    \item Para acceder a los elementos hay que iterar la lista
  \end{itemize}
\end{frame}

\begin{frame}{Listas enlazadas}
  \lstinputlisting[caption=Vectores en C++, label={lst:listing-cpp}, language=C++, basicstyle=\fontsize{8}{9}\selectfont]{list\_sample.cpp}
\end{frame}

\begin{frame}{Listas enlazadas vs vectores}
  \begin{itemize}
    \item Los vectores son almacenados en espacios de memoria contigua, las listas no
      \pause
    \item Por ello, la inserci\'on en una posici\'on cualquiera es lenta para un vector, pero r\'apida para una lista
      \pause
    \item An\'alogamente, la remoci\'on en una posici\'on cualquiera es lenta para un vector, pero r\'apida para una lista
      \pause
    \item El uso de uno u otro depende de lo que se necesite
  \end{itemize}
\end{frame}

\begin{frame}{Hash tables}
  \begin{itemize}
    \item Tambi\'en llamados diccionarios o maps, son contenedores que asocian dos valores
      \pause
    \item Se pueden pensar como una generalizaci\'on de los arreglos
      \pause
    \item En un arreglo, cada elemento tiene un \'indice que es un n\'umero entero
      \pause
    \item En un map, cada elemento tiene un \'indice que puede ser cualquier tipo de dato
      \pause
    \item Se puede pensar como una agenda telef\'onica, en la que se indexa (o busca) por nombres y apellidos y no por \'indices enteros
      \pause
  \end{itemize}
\end{frame}

\begin{frame}{Hash tables}
  \lstinputlisting[caption=Vectores en C++, label={lst:listing-cpp}, language=C++, basicstyle=\fontsize{8}{9}\selectfont]{map\_sample.cpp}
\end{frame}

\begin{frame}{Hash tables}
  \begin{itemize}
    \item De manera an\'aloga a los sets, los maps preservan el orden de los elementos en base al \'indice
      \pause
    \item Mantener el orden implica m\'as esfuerzo computacional, por tanto se pierde un poco de eficiencia
      \pause
    \item Si no se desea mantener el orden se puede usar \texttt{unordered\_map}
  \end{itemize}
\end{frame}

\begin{frame}{Documentaci\'on adicional}
  \begin{itemize}
    \item Documentaci\'on oficial: https://en.cppreference.com/w/cpp/container
    \item Geek for geeks: https://www.geeksforgeeks.org/containers-cpp-stl/
    \item Curso de geek for geeks: https://www.geeksforgeeks.org/courses/cpp-programming-basic-to-advanced
  \end{itemize}
\end{frame}

\subsection{Estructuras de Datos Avanzadas}
\begin{frame}{Contenido}
\tableofcontents
\end{frame}

\begin{frame}{Estructuras de Datos Avanzadas}
  \begin{itemize}
    \item Para efectos de este curso, entendemos como estructuras de datos avanzadas a aquellas que no son implemendatas por la librer\'ia estandar de C++ (STL), sino que deben ser implementadas
      \pause
    \item Al implementar una estructura de datos debemos pensar en sus diversas operaciones: inicializaci\'on, inserci\'on, remoci\'on, b\'usqueda, etc. Las operaciones que soporte cada estructura de datos depende de su dise\~no
      \pause
    \item Para tener una estructura de datos eficiente, sus operaciones deben ser eficientes
  \end{itemize}
\end{frame}

\begin{frame}{Estructuras de Datos Avanzadas}
  \begin{itemize}
    \item Una situaci\'on usual es tener alguna operaci\'on computacionalmente costosa (lenta), pero el resto de operaciones eficientes. 
      \pause
    \item Si la operaci\'on costosa es utilizada pocas veces y las operaciones eficientes muchas veces, en general se tendr\'a una buena base para un algoritmo
  \end{itemize}
\end{frame}

\begin{frame}{\'Arboles}
  \begin{itemize}
    \item Es una estructura de datos jer\'arquica, conformado por varios nodos
      \pause
    \item Cada nodo tiene un nodo \textit{padre} y guarda un valor
      \pause
    \item El nodo \textit{ra\'iz} es aquel que no tiene ning\'un padre. 
      \pause
    \item Usualmente se tiene solo una referencia al nodo padre y se puede explorar el \'arbol a partir de este
      \pause
    \item Existen varias formas de implementar un \'arbol, cada una con sus ventajas y desventajas, tanto en t\'erminos de memoria como de tiempo
      \pause
    \item Se conoce como \textit{altura} del \'arbol al n\'umero de niveles que tiene el \'arbol
  \end{itemize}
\end{frame}

\begin{frame}{\'Arboles}
  \centering{
    \begin{tikzpicture}[
        level distance=1.5cm,
        sibling distance=1.5cm,
        every node/.style={circle, draw}
      ]
      % Nodes
      \node {1}
      child {node {2}
      child {node {1}}
      child {node {6}}
      }
      child {node[right=of current axis, xshift=2cm, circle, draw] {7}
      child {node {5}}
      child {node {8}}
      child {node {3}}
      };
    \end{tikzpicture}
    }
\end{frame}

\begin{frame}{\'Arboles}
  \lstinputlisting[caption=Vectores en C++, label={lst:listing-cpp}, language=C++, basicstyle=\fontsize{8}{9}\selectfont]{tree\_sample.cpp}
\end{frame}

\begin{frame}{\'Arboles}
  \lstinputlisting[caption=Vectores en C++, label={lst:listing-cpp}, language=C++, basicstyle=\fontsize{8}{9}\selectfont]{tree\_sample2.cpp}
\end{frame}

\begin{frame}{\'Arboles binarios}
  \begin{itemize}
    \item Son \'arboles en los que cada nodo tiene a lo mucho dos hijos
      \pause
    \item Se dice que un \'arbol est\'a completo cuando todos los niveles est\'an llenos, excepto el \'ultimo
      \pause
    \item Cuando el \'arbol est\'a completo se cumple la siguiente relaci\'on entre la altura y el n\'umero de nodos del \'arbol $h = \left\lceil \log n \right\rceil $
      \pause
    \item \'Arboles completos pueden ser f\'acilmente representados con un arreglo; si el \'arbol no est\'a completo es mejor usar un map
  \end{itemize}
\end{frame}

\begin{frame}{\'Arboles binarios}
  \centering{
    \begin{tikzpicture}[
        level distance=1cm,
        sibling distance=2.5cm,
        every node/.style={circle, draw}
      ]
      % Nodes
      \node {1}
      child {node {2}
      child {node {1}}
      child {node {6}}
      }
      child {node[right=of current axis, xshift=2cm, circle, draw] {7}
      child {node {5}}
      child {node {8}}
      };
    \end{tikzpicture}
    }
\end{frame}

\begin{frame}{\'Arboles binarios}
  \lstinputlisting[caption=Arbol binario, label={lst:listing-cpp}, language=C++, basicstyle=\fontsize{8}{9}\selectfont]{binary\_tree\_sample.cpp}
\end{frame}

\begin{frame}{\'Arboles binarios}
  \lstinputlisting[caption=Arbol binario, label={lst:listing-cpp}, language=C++, basicstyle=\fontsize{8}{9}\selectfont]{binary\_tree\_sample2.cpp}
\end{frame}

\begin{frame}{\'Arboles de b\'usqueda binarios}
  \begin{itemize}
    \item Son \'arboles binarios tal que para cualquier nodo, el hijo de la izquierda es menor que el nodo padre, y el hijo de la derecha es mayor que el nodo padre
      \pause
    \item Se puede generalizar la noci\'on al integrar una relaci\'on de orden entre el nodo padre y los nodos hijos
  \end{itemize}
\end{frame}

\begin{frame}{\'Arboles de b\'usqueda binarios}
  \centering{
    \begin{tikzpicture}[
        level distance=1cm,
        sibling distance=2.5cm,
        every node/.style={circle, draw}
      ]
      % Nodes
      \node {5}
      child {node {2}
      child {node {1}}
      child {node {4}}
      }
      child {node[right=of current axis, xshift=2cm, circle, draw] {7}
      child {node {6}}
      child {node {8}}
      };
    \end{tikzpicture}
    }
\end{frame}

\begin{frame}{\'Arboles de b\'usqueda binarios}
  \lstinputlisting[caption=Arbol binario, label={lst:listing-cpp}, language=C++, basicstyle=\fontsize{8}{9}\selectfont]{bst\_sample.cpp}
\end{frame}

\end{document}
